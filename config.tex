% Pacotes
% Principais
\usepackage[T1]{fontenc}		% Selecao de codigos de fonte.
\usepackage[utf8]{inputenc}		% Codificacao do documento (conversão automática dos acentos)
\usepackage{indentfirst}		% Indenta o primeiro parágrafo de cada seção.
\usepackage{color}				% Controle das cores
\usepackage{graphicx}			% Inclusão de gráficos
\usepackage{subfigure}
\usepackage{microtype} 			% para melhorias de justificação
\usepackage{titlesec}           % para definir o formato do título
\usepackage{enumitem}           % Para realizar enumerações com ítens


% --------------
\usepackage{abntex2unifei}   % Adequações à Unifei. Usando localização relativa.
% --------------

% Citações
\usepackage[brazilian,hyperpageref]{backref}	 % Paginas com as citações na bibl
\usepackage[alf,abnt-emphasize=bf]{abntex2cite}  % Citações padrão ABNT, com referências em negrito



\usepackage{lipsum}				% para geração de dummy text




% Configurações de pacotes

% Configurações do pacote backref
% Usado sem a opção hyperpageref de backref
\renewcommand{\backrefpagesname}{Citado na(s) página(s):~}
% Texto padrão antes do número das páginas
\renewcommand{\backref}{}
% Define os textos da citação
\renewcommand*{\backrefalt}[4]{
	\ifcase #1 %
		Nenhuma citação no texto.%
	\or
		Citado na página #2.%
	\else
		Citado #1 vezes nas páginas #2.%
	\fi}%



% Configurações de aparência do PDF final

% alterando o aspecto da cor azul
\definecolor{blue}{RGB}{41,5,195}

% Espaçamentos entre parágrafo
% O tamanho do parágrafo é dado por:
\setlength{\parindent}{1.3cm}
% Controle do espaçamento entre um parágrafo e outro:
\setlength{\parskip}{0.3cm}  % tente também \onelineskip



%%%% CONFIGURAÇÕES DO TRABALHO %%%%
% Mude estes comandos para adequar seu trabalho a eles.

% Informações de dados para CAPA e FOLHA DE ROSTO
\titulo{Uma aplicação de aprendizado de máquina na classificação de galaxias}
\autor{Rafael J. Rangel}
\local{Itajubá}
\data{2020}
\instituicao{Universidade Federal de Itajubá - UNIFEI}
\tipotrabalho{Trabalho de Conclusão de Curso}
\orientador{Hektor Sthenos Alves Monteiro}
% O preambulo deve conter o tipo do trabalho, o objetivo, 
% o nome da instituição e a área de concentração.
%Estas informações serão impressas, por exemplo, na sua folha de rosto.
\preambulo{Trabalho de Conclusão de Curso apresentado à Universidade Federal de Itajubá como requisito para obtenção do grau de bacharel em Física.}


% Configurações gerais do PDF gerado (cor dos links, etc)
% Parte destes dados são gerados automaticamente.
\makeatletter
\hypersetup{
     	%pagebackref=true,
		pdftitle={\@title}, 
		pdfauthor={\@author},
    	pdfsubject={\imprimirpreambulo},
	    pdfcreator={trabalho},
		pdfkeywords={abntex2}{abntex2unifei}{latex}{texto}, 
		colorlinks=true,    % Links coloridos. Falso para criar caixas ao redor
        linkcolor=black,    % Cor de links internos (seções, etc)
    	citecolor=green,    % Cor de links para a bibliografia
    	filecolor=magenta,  % Cor de links para arquivos
		urlcolor=blue,      % Cor de links para URLs da web
		bookmarksdepth=4
}
\makeatother
